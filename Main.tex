\documentclass{beamer}
\usepackage{subcaption}
\usepackage{float}
\usepackage{setspace}
\usepackage{wrapfig}
\usepackage{caption}
\captionsetup[figure]{labelformat=empty}
\usetheme[pageofpages=of,% String used between the current page and the
                         % total page count.
          bullet=circle,% Use circles instead of squares for bullets.
          titleline=true,% Show a line below the frame title.
          alternativetitlepage=true,% Use the fancy title page.
          titlepagelogo=logo-polito,% Logo for the first page.
          watermark=watermark-polito,% Watermark used in every page.
          watermarkheight=100px,% Height of the watermark.
          watermarkheightmult=4,% The watermark image is 4 times bigger
                                % than watermarkheight.
          ]{Theme}
\usepackage{lmodern}
\author{\small By\\Bharat Bhushan (153100048)}
\title{{{Thermoelastic fracture problems using Extended Finite Element Method}}}
\institute{\small Under the guidance of\\Prof. Salil S. Kulkarni\\ \vspace{5pt}Department of Mechanical Engineering, IIT Bombay}
\date{\vspace{-5pt}October 18, 2016}

\begin{document}
%%%%%%%%%%%%%%%%%%%%%%%%%%%%%%%%%%%%%%%%%%%%%%%%%%%%%%%%%%%%%%%%%%%%5
%%%% TITLE PAGE %%%%%%%%%%%%%%%%%%%%%%%%%%%%%%%%%%%%%%%%%%%%%%%%%%%%
%%%%%%%%%%%%%%%%%%%%%%%%%%%%%%%%%%%%%%%%%%%%%%%%%%%%%%%%%%%%%%%%%%%%%
\begin{frame}[t,plain]
\titlepage
\end{frame}
%%%%%%%%%%%%%%%%%%%%%%%%%%%%%%%%%%%%%%%%%%%%%%%%%%%%%%%%%%%%%%%%%%%%%%
%%%% content %%%%%%%%%%%%%%%%%%%%%%%%%%%%%%%%%%%%%%%%%%%%%%%%%%%%%%%
%%%%%%%%%%%%%%%%%%%%%%%%%%%%%%%%%%%%%%%%%%%%%%%%%%%%%%%%%%%%%%%%%%%%

%%%%%%%%%%%%%%%%%%%%%%%%%%%%%%%%%%%%%%%%%%%%%%%%%%%%%%%%%%%%%%%%%
\begin{frame}[t,fragile]{Outline}
    \begin{itemize}
        \item Introduction  
        \item Motivation 
        \item Literature Survey 
        \item Work done 
        \item Problem definition 
        \item Conclusion 
    \end{itemize}
\end{frame}
%%%%%%%%%%%%%%%%%%%%%%%%%%%%%%%%%%%%%%%%%%%%%%%%%%%%%%%%%%%%%%%%%
\begin{frame}[t,fragile]{Introduction: Thermo-elastic Fracture Problems}
    \vspace{-.3cm}
    \begin{itemize}
        \item Due to heat transfer, temperature field is set up in the material which induces thermal stresses in the body.
\item These stresses becomes large in the vicinity of a discontinuity i.e. crack tip. If temperature variation is sufficiently large, it can lead to failure.
        \item Applications: Nuclear power plants, cylinder-nozzle
 intersection in pressure vessels, aerodynamic heating of high-speed aircraft, ultra fast pulse lasers etc.
   \end{itemize}
   \vspace{-.5cm}
\begin{figure}[H]
    \hspace{.7cm}
      \begin{subfigure}{0.45\textwidth}
    \centering
 \includegraphics[scale=.1]{cyl.png}
 \caption{\tiny{The cylinder-nozzle intersection. (www.knowledge.autodesk.com)}}
 \label{cyl}
 \end{subfigure}
\begin{subfigure}{0.45\textwidth}
    \centering
 \includegraphics[scale=.1]{fail.jpg}
 \caption{\tiny{Cracked head of baffle bolt of Belgian Nuclear Reactor.(www.miningawareness.wordpress.com)}}
 \label{fail}
 \end{subfigure}
 \end{figure}
 \vspace{-.3cm}
   \tiny
   \hspace{15pt}
   \textbf{source}: Tian, X., Shen. (2006). A direct finite element method study of generalized thermoelastic problems. \\
   \vspace{-7pt}
   \hspace{15pt}
   \emph{International Journal of Solids and Structures}, 43(7), 2050-2063.
\end{frame}
%%%%%%%%%%%%%%%%%%%%%%%%%%%%%%%%%%%%%%%%%%%%%%%%%%%%%%%%%%%%%%%%%o 
\begin{frame}[t,fragile]{Why thermal load on crack is important?}
    \vspace{-.3cm}
    \footnotesize
\begin{itemize}
    \item Atkinson has solved the Dirichlet problem for Laplace's equation on a pie shaped region as $u(x,y)= r^{\frac{\pi}{\phi}}\sin\alpha\theta,\  r>0,\ 0<\theta<\phi$
        \begin{itemize}
                \footnotesize
    \item If $0<\phi<\pi$:
        The first partial derivative of u with respect to x and y remains continuous as we approach towards the origin. 
    \item If $\pi<\phi<2\pi$:
        The first derivative u with respect to x and y are not continuous as (x,y) approaches the origin. 
    \end{itemize}
    \item When $\phi=2\pi$, the problem becomes a crack problem and displacement and derivative of displacement vary as $u \propto r^{\frac{1}{2}}$ and $u'\propto  r^{-\frac{1}{2}}$ respectively. 
    \item As thermo-elastic problems are also governed by Laplace's equation, temperature will vary as $r^{\frac{1}{2}}$ and heat fluxes will be unbounded at the crack tip. 
\end{itemize}
  \tiny
  \vspace{10pt}
  \hspace{10pt}
   \textbf{source}: Atkinson, K. E. (1997).
    \emph{The numerical solution of \\
  \hspace{10pt}
    integral equations of the second kind} (Vol. 4). \\
  \hspace{10pt}
    Cambridge university press.
\begin{wrapfigure}{r}{0.4\textwidth}
    \centering
    \vspace{-50pt}
    \includegraphics[width=.3\textwidth]{pie.png}
    \caption{\footnotesize Pie-shaped region.}
    \label{pie}
\end{wrapfigure}
 
\end{frame}
%%%%%%%%%%%%%%%%%%%%%%%%%%%%%%%%%%%%%%%%%%%%%%%%%%%%%%%%%%%%%%%%%%%%
\begin{frame}[t,fragile]{Methods of Solution}
    \vspace{-.4cm}
    \begin{itemize}
        \item Our main concern in case of thermally loaded fracture problems is to find the stress intensity factors. Methods for calculating stress intensity factors in FEM can be divided in two categories,
            \begin{itemize}
                \item Substitution method 
                \item Energy method 
            \end{itemize}
        \item In substitution method, we can get the SIF's as:
            \footnotesize
            $$ u=\frac{1+\nu}{4E}\sqrt{\frac{2r}{\pi}}\left\{ K_I\left[ (2\kappa -1)\cos \frac{\theta}{2}-\cos \frac{3\theta}{2} \right]+K_{II}\left[ (2\kappa +3)\sin\frac{\theta}{2}+\sin\frac{3\theta}{2} \right] \right\}$$
            \normalsize
A similar equation exist for $v$ also. 
        \item Using energy method, SIF's can be calculated as$^\ast$ 
            \footnotesize
$$G=-\frac{d\Pi}{da}\ ,\ \ \ \ \ \  
\Pi=-\frac{1}{2}{u}^T[K]{u}+\frac{1}{2}\int{\varepsilon_0}^T[D]{\varepsilon_0}dV$$
 
    \end{itemize}
    \vspace{-.1cm}
   \tiny
   \hspace{15pt}
   $^\ast$\textbf{source}:Hellen, T. K., \& Cesari, F. (1979). On the solution of the centre cracked plate with a quadratic thermal\\ 
   \vspace{-7pt}
   \hspace{15pt}
   gradient.\emph{Engineering Fracture Mechanics}, 12(4), 469-478.
\end{frame} 
%%%%%%%%%%%%%%%%%%%%%%%%%%%%%%%%%%%%%%%%%%%%%%%%%%%%%%%%%%%%%%%%%%%
\begin{frame}[t,fragile]{}
\end{frame}
%%%%%%%%%%%%%%%%%%%%%%%%%%%%%%%%%%%%%%%%%%%%%%%%%%%%%%%%%%%%%%%%%%%%
\begin{frame}[t,fragile]{Objectives}
    \begin{itemize}
        \item Finite Element Formulation of thermo-elastic problems
        \item Computer implementation of the FEM model in MATLAB
        \item Solving the crack problems with thermal loading
        \item Application of the Extended Finite Element Method 
    \end{itemize}
\end{frame}
%%%%%%%%%%%%%%%%%%%%%%%%%%%%%%%%%%%%%%%%%%%%%%%%%%%%%%%%%%%%%%%%%%%
\begin{frame}[t,fragile]{Finite Element Formulation of Thermo-elasticity}
\begin{itemize}
    \item Since problems such as coupled structural-diffusion problems are governed by similar equations, we will be able to analyze all such coupled problems by only changing the field variable of the thermo-mechanical model.
\item In semi-coupled analysis we neglect the effect of displacements on temperature field. 
\item The thermo-elastic constitutive relation for plane stress is given by:
    \small
\begin{align*}
    \begin{Bmatrix}
        \sigma_{x}\\ \sigma_{y}\\ \tau_{xy} 
    \end{Bmatrix} =\frac{E}{(1-\nu^2)}
    \begin{bmatrix}
        1 & \nu & 0 \\ \nu & 1 & 0 \\ 0 & 0 & 1-\nu 
    \end{bmatrix}
    \begin{Bmatrix}
        \varepsilon_{x}-\alpha\Delta T \\ \varepsilon_{y}-\alpha \Delta T \\ \varepsilon_{xy} 
    \end{Bmatrix}
\end{align*}
\end{itemize}

\end{frame}
%%%%%%%%%%%%%%%%%%%%%%%%%%%%%%%%%%%%%%%%%%%%%%%%%%%%%%%%%%%%%%%
\begin{frame}[t,fragile]{Constitutive relations in one dimensions}
    \begin{itemize}
        \item we can get the total strain as 
\begin{align}
    \varepsilon_{ij}&=\varepsilon_{ij}^{(M)}+\varepsilon_{ij}^{(T)}\nonumber \\
    &=\frac{1+\nu}{E}\sigma_{ij}-\frac{\nu}{E}\sigma_{kk}\delta_{ij}+\alpha(T-T_0)\delta_{ij}
\end{align}
      \item  For isotropic case equation \ref{4} can be inverted to get the stresses 
\begin{equation}
    \sigma_{ij}=\lambda\varepsilon_{kk}\delta_{ij}+2\mu\varepsilon_{ij}-(3\lambda+2\mu)\alpha(T-T_0)\delta_{ij}
\end{equation}
where $\lambda$ and $\mu$ are Lame constants and are given as 
\begin{align}\lambda=\frac{E\nu}{(1+\nu)(1-2\nu)}~ ~,~ ~ ~ ~ ~\mu=\frac{E}{2(1+\nu)}\end{align}
    \end{itemize}
\end{frame}
%%%%%%%%%%%%%%%%%%%%%%%%%%%%%%%%%%%%%%%%%%%%%%%%%%%%%%%%%%%%%%%%%%%
\begin{frame}[t,fragile]{Constitutive relations in two dimensions}
    \begin{itemize}
        \item Thus writing common constitutive relation for both plane strain and plane stress using equations   
\begin{align}
\begin{Bmatrix}
    \sigma_x\\ \sigma_y\\ \tau_{xy}   
\end{Bmatrix}=
\begin{bmatrix}
    c_{11}&c_{12}&0\\c_{12}&c_{22}&0\\0&0&c_{66}
\end{bmatrix}
\begin{Bmatrix}
    \varepsilon_x-\alpha T\\ \varepsilon_y-\alpha T\\ \varepsilon_{xy}
\end{Bmatrix} 
\end{align}
where for plane strain 
$$c_{11}=c_{22}=\frac{E(1-\nu)}{(1+\nu)(1-2\nu)}~ ~ ,~ ~ c_{12}=\frac{E\nu}{(1+\nu)(1-2\nu)}~ ~,~  ~ c_{66}=\frac{E}{1+\nu}$$
and for plane stress 
$$c_{11}=c_{22}=\frac{E}{(1-\nu^2)}~ ~ ,~ ~ c_{12}=\frac{E\nu}{(1-\nu^2)}~ ~,~  ~ c_{66}=\frac{E}{1+\nu}$$
    \end{itemize}
\end{frame}
    %%%%%%%%%%%%%%%%%%%%%%%%%%%%%%%%%%%%%%%%%%%%%%%%%%%%%%%%
   \begin{frame}[t,fragile]{Finite element model}
    \begin{itemize}
        \item The governing equations of the thermo-elasticity is given by:
            \bgroup
            \tiny
            \begin{align*}
    \frac{\partial}{\partial x}\left[c_{11}\frac{\partial u}{\partial x}+c_{12}\frac{\partial v}{\partial y}\right]+&\frac{\partial}{\partial y}\left[c_{66}\left(\frac{\partial u}{\partial y}+\frac{\partial v}{\partial x}\right)\right]-(c_{11}+c_{12})\alpha\frac{\partial T}{\partial x}-f_x   =0 \\
    \frac{\partial}{\partial x}\left[c_{66}\left(\frac{\partial u}{\partial y}+\frac{\partial v}{\partial x}\right)\right]+&\frac{\partial}{\partial y}\left[c_{12}\frac{\partial u}{\partial x}+c_{22}\frac{\partial v}{\partial y}\right]-(c_{11}+c_{12})\alpha\frac{\partial T}{\partial y}-f_y=0\\
    &\ \ \ \ \ \ \ k\left( \frac{\partial^2 T}{\partial x^2}+\frac{\partial^2 T}{\partial y^2} \right)=q
\end{align*}
\egroup
where k is the thermal conductivity, T is temperature and q is the heat source.
    \end{itemize}
\end{frame}
%%%%%%%%%%%%%%%%%%%%%%%%%%%%%%%%%%%%%%%%%%%%%%%%%%%%%%%%%%%%%%%%%%%%%%
\begin{frame}[t,fragile]{Constitutive relations in two dimensions}
    \begin{itemize}
        \item Approximating u,v and T over a typical finite element $\Omega^e$ by the expression \cite{reddy}.
\begin{align}
    u(x,y)=\sum_{i=1}^nN_i (x,y)u_i\label{u}\\
    v(x,y)=\sum_{i=1}^nN_i (x,y)v_i\label{v}\\
    T(x,y)=\sum_{i=1}^nN_i (x,y)T_i\label{t}
\end{align}
    \end{itemize}
\end{frame}
%%%%%%%%%%%%%%%%%%%%%%%%%%%%%%%%%%%%%%%%%%%%%%%%%%%%%%%%%%%%%%%%%%%
\begin{frame}[t,fragile]{weak form}
    \begin{itemize}
            \tiny
        \item \begin{align}
     -\int_{\Omega}^{}\left[ c_{11}\frac{\partial N_i}{\partial x}\frac{\partial N_j}{\partial x}u_j+c_{12}\frac{\partial N_i}{\partial x}\frac{\partial N_j}{\partial y}v_j\right]dxdy+\int_{\Omega}^{}\left[c_{66}\frac{\partial N_i}{\partial y}\frac{\partial N_j}{\partial y}u_jdxdy+\frac{\partial N_i}{\partial y}\frac{\partial N_j}{\partial x}v_j\right]dxdy\nonumber\\ -\int_{\Omega}^{}\frac{\partial N_i}{\partial x}\beta N_jT_j dxdy +\int_{\Omega}^{}N_if_x
     dxdy+\int_{\Gamma}^{}N_i\vec{t}dx=0\label{24}\end{align}
 \begin{align}
     -\int_{\Omega}^{}\left[ c_{11}\frac{\partial N_i}{\partial y}\frac{\partial N_j}{\partial y}v_j+c_{12}\frac{\partial N_i}{\partial y}\frac{\partial N_j}{\partial x}u_j\right]dxdy+\int_{\Omega}^{}\left[c_{66}\frac{\partial N_i}{\partial x}\frac{\partial N_j}{\partial x}v_jdxdy+\frac{\partial N_i}{\partial x}\frac{\partial N_j}{\partial y}u_j\right]dxdy\nonumber\\ -\int_{\Omega}^{}\frac{\partial N_i}{\partial y}\beta N_jT_j dxdy+\int_{\Omega}^{}N_if_y dxdy+\int_{\Gamma}^{}N_i\vec{t}dy=0\label{25}\\
        k\int_{\Omega}\left( \frac{\partial N_i}{\partial x}\frac{\partial N_j}{\partial x}+\frac{\partial N_i}{\partial y}\frac{\partial N_j}{\partial y} \right)T_jdxdy-\int_{\Omega}^{}N_iN_jqdxdy=\int_{\Gamma}^{}N_i\bar{Q}ds& \label{26} \end{align}
Neglecting the body forces, above equations can be written in matrix form as follows\\ 
    \begin{align}
\begin{bmatrix}
    K_{11} & K_{12} \\
    0 & K_{22}
\end{bmatrix}
\begin{Bmatrix}
    U^e\\ T^e
\end{Bmatrix}=
\begin{Bmatrix}
    F\\ Q
\end{Bmatrix}
\end{align} 
    \end{itemize}
\end{frame}
%%%%%%%%%%%%%%%%%%%%%%%%%%%%%%%%%%%%%%%%%%%%%%%%%%%%%%%%%%%%%%%%%%%
\begin{frame}[t,fragile]{weak form}
    \tiny
    \begin{itemize}
        \item Where $U^e=[u~ ~ v]^T$ and $T^e$ are the unknown variables to be found and $K_{11}$, $K_{12}$ and $K_{22}$ are the stiffness matrices which are defined as below:\\  
\begin{align*}
K_{11}&=\int_{\Omega}^{}
\begin{bmatrix}
    c_{11}\frac{\partial N_i}{\partial x}\frac{\partial N_j}{\partial x}+c_{66}\frac{\partial N_i}{\partial y}\frac{\partial N_j}{\partial y} & c_{12}\frac{\partial N_i}{\partial x}\frac{\partial N_j}{\partial y}+c_{66}\frac{\partial N_i}{\partial y}\frac{\partial N_j}{\partial x}\\
    c_{12}\frac{\partial N_i}{\partial y}\frac{\partial N_j}{\partial x}+\frac{\partial N_i}{\partial x}\frac{\partial N_j}{\partial y}& c_{11}\frac{\partial N_i}{\partial y}\frac{\partial N_j}{\partial y}+c_{66}\frac{\partial N_i}{\partial x}\frac{\partial N_j}{\partial y}
\end{bmatrix}dxdy\\\\
K_{12}&=-\beta\int_{\Omega}^{}
\begin{bmatrix}
    \frac{\partial N_i}{\partial x}N_j \\ \frac{\partial N_i}{\partial y}N_j
\end{bmatrix}dxdy\\\\
K_{22}&=\int_{\Omega}^{}K\left( \frac{\partial N_i}{\partial x}\frac{\partial N_j}{\partial x}+\frac{\partial N_i}{\partial y}\frac{\partial N_j}{\partial y} \right)dxdy\\\\
F&= \int_{\Omega} N_i \begin{bmatrix} f_x\\f_y \end{bmatrix}dxdy~ ~ ~ ~ ~ ~ ~ ~ ~  ~ ~~ ~ ~ ~
Q=\int_{\Gamma}N_i\bar{Q}ds~ ~ ~ ~ and~ ~ ~ ~ U=[u~ ~ v]^T
\end{align*}
where the constants $c_{11}, c_{12}, c_{66}$
for plane strain is given by:
$$c_{11}=c_{22}=\frac{E(1-\nu)}{(1+\nu)(1-2\nu)}~ ~ ,~ ~ c_{12}=\frac{E\nu}{(1+\nu)(1-2\nu)}~ ~,~  ~ c_{66}=\frac{E}{1+\nu}$$
and for plane stress is given by: 
$$c_{11}=c_{22}=\frac{E}{(1-\nu^2)}~ ~ ,~ ~ c_{12}=\frac{E\nu}{(1-\nu^2)}~ ~,~  ~ c_{66}=\frac{E}{1+\nu}$$
\end{itemize}
\end{frame}
%%%%%%%%%%%%%%%%%%%%%%%%%%%%%%%%%%%%%%%%%%%%%%%%%%%%%%%%%%%%%%%%%%%%%%
\begin{frame}[t,fragile]{weak form}
    \tiny
         $$ N=\begin{bmatrix}N_1 &0&N_2 &0&\dots&N_n &0\\0&N_1 &0&N_2 &\dots&0&N_n \end{bmatrix}~and ~N^{\theta}=[N_1 ~\dots~N_n ] 
$$
Here $N$ and $N^{\theta}$ are the shape functions for displacement and temperature fields respectively. So the approximation of fields withing one element in matrix form can be written as:
$$ \begin{Bmatrix}u\\v\end{Bmatrix}=N\ U^e~ ~ ~ ~and ~ ~~ T=N^{\theta}\ T^e
$$
If we define the matrices $[B]$ and $[B^{\theta}]$ as follows:
    \begin{align}
    [B]=\begin{bmatrix}
        \frac{\partial N_1}{\partial x}&0&\frac{\partial N_2}{\partial x}&0&\dots&\frac{\partial N_n}{\partial x}&0\\
        0&\frac{\partial N_1}{\partial y}&0&\frac{\partial N_2}{\partial y}&\dots&0&\frac{\partial N_n}{\partial y}\\
\frac{\partial N_1}{\partial y}&\frac{\partial N_1}{\partial x}&
\frac{\partial N_2}{\partial y}&\frac{\partial N_2}{\partial x}&\dots&
\frac{\partial N_n}{\partial y}&\frac{\partial N_n}{\partial x}
    \end{bmatrix}~ ~ and ~ ~    [B^{\theta}]=\begin{bmatrix}
        \frac{\partial N_1}{\partial x}&\frac{\partial N_2}{\partial x}&\dots&\frac{\partial N_n}{\partial x}\\
        \frac{\partial N_1}{\partial y}&\frac{\partial N_2}{\partial y}&\dots&\frac{\partial N_n}{\partial y}\\
    \end{bmatrix}
    \end{align}
The strains $\{\varepsilon\}$ and temperature gradients $\{\theta'\}$ can be written as follows:
    \begin{align}
    \left\{ \varepsilon \right\}=\left[ B \right]\{ U^{(e)} \} ~ ,~ ~ ~ ~ ~ ~ ~ ~ ~  ~ \{\theta'\}=\left[ B^{\theta} \right]\{ T^{(e)} \} 
\end{align}   
\end{frame} 

%%%%%%%%%%%%%%%%%%%%%%%%%%%%%%%%%%%%%%%%%%%%%%%%%%%%%%%%%%%%%%%%%%%%%%%%%
\begin{frame}[t,fragile]{weak form}
    \tiny
\begin{itemize}
    \item These expressions can also be written in the matrix as described by Tian \cite{tian}:
\begin{align}
    [K_{11}^e]&=\int_{\Omega}[B]^T[C][B]dxdy\\
    [K_{12}^e]&=\int_{\Omega}[B]^T[\beta][N^{\theta}]dxdy\\
    [K_{22}^e]&=\int_{\Omega}[B^{\theta}]^T[K][B^{\theta}]dxdy
\end{align}
\begin{align}
    \{F\}&=\int_\Gamma [N]^T{\bar{t}}ds\\
    \{Q\}&=\int_\Gamma [N^{\theta}]^T\bar{Q}ds
    \label{}
\end{align}
\end{itemize}
\end{frame}
%%%%%%%%%%%%%%%%%%%%%%%%%%%%%%%%%%%%%%%%%%%%%%%%%%%%%%%%%%%%%%%%%%%%%%%%
\begin{frame}[t,fragile]{Computer implementation}
\begin{wrapfigure}{r}{0.5\textwidth}
  \begin{center}
      \vspace{-1.3cm}
\includegraphics[width=0.48\textwidth]{flow_chart.pdf}
  \end{center}
\caption{Flowchart showing the steps of the FEM program}
\end{wrapfigure}
We have developed the 2-dimensional Finite Element Program and the elements used is quadrilateral elements i.e. Q4 elements. We transformed the quadrilateral element of a mesh to the master element $\hat{\Omega}$ (fig. and used $2\times 2$ Gauss quadrature rule for numerical integration.

\end{frame}
%%%%%%%%%%%%%%%%%%%%%%%%%%%%%%%%%%%%%%%%%%%%%%%%%%%%%%%%%%%%%%%%%%%%%
\begin{frame}[t,fragile]{Coupled Thermo-elasticity Problems}
    \begin{itemize}
        \item Coupled thermoelasticity problems have become very important in recent years because of its use in various industries.
        \item Some application areas include 
            \begin{itemize}
                \item aerodynamic heating of high speed air crafts as shown in figure (a). 
            \item the nuclear reactors where very high-temperatures and temperature gradients are developed as shown in figure (b).
            \item the ultra fast pulse lasers which is used for
micro-machining.
            \item non destructive detection.
            \item natural characterisation etc 
            \end{itemize}
    \end{itemize}
    \vspace{-.5cm}
    \begin{figure}[H]
      \hspace{.5cm}
\begin{subfigure}{0.45\textwidth}
  \vspace{.5cm}
    \centering
 \includegraphics[scale=.1]{hyper.jpg}
 \caption{\tiny{Hyper-X vehicle at Mach 7.Source: www.dfrc.nasa.gov}}
 \label{1}
 \end{subfigure}
 \begin{subfigure}{0.45\textwidth}
    \centering
 \includegraphics[scale=.05]{reator.png}
 \caption{\tiny{High temperature reactor. Source: https://commons.wikimedia.org}}
 \label{2}
 \end{subfigure}
\label{fig:caption}
\end{figure}
\end{frame}
%%%%%%%%%%%%%%%%%%%%%%%%%%%%%%%%%%%%%%%%%%%%%%%%%%%%%%%%%%%%%%%%%%%%%
\begin{frame}[t,fragile]{Extended Finite Element Method}
    \begin{itemize}
\item Present work describes the application of the eXtended Finite Element Method (XFEM) in coupled thermoelastic problems.
\item In classical finite element method, mesh should conform to the boundaries of discontinuity for accurate modeling of the problem. So remeshing has to be done every time the crack grows. 
\item An alternate method is to enrich the polynomial approximation with functions which can model the discontinuities. 
    \footnotesize
    \begin{align*}
    u^h&=\sum_i N_i(x)u_i+\sum_{j\in J} N_j(x) h(x)a_j+\sum_{k\in K} N_k(x)\left( \sum_{l=1}^{4}\gamma_l(x)b_{kl} \right) \\
    v^h&=\sum_i N_i(x)v_i+\sum_{j\in J} N_j(x) h(x)c_j+\sum_{k\in K} N_k(x)\left( \sum_{l=1}^{4}\gamma_l(x)d_{kl} \right) 
\end{align*}
\end{itemize}
\end{frame}
%%%%%%%%%%%%%%%%%%%%%%%%%%%%%%%%%%%%%%%%%%%%%%%%%%%%%%%%%%%%%%%%%%%%%%%%%%%%%
\begin{frame}[t,fragile]{\ldots Extended Finite Element Method }
    \begin{itemize}
        \item 
            \footnotesize \begin{align*}
\gamma=\left[ \sqrt{r}\cos \left( \frac{\theta}{2} \right), \sqrt{r}\sin\left( \frac{\theta}{2} \right),\sqrt{r}\sin\left( \frac{\theta}{2} \right)\sin(\theta),\sqrt{r}\cos\left( \frac{\theta}{2} \right)\sin(\theta)\right] 
\end{align*}
\item As can be seen above the third singular function $\gamma_3$ is the only enrichment function which is discontinuous across the crack. Thus the discontinuity of the displacement field at $\theta=\pm \pi$ in the singular zone is only modeled by $\gamma_3$ on the elements containing the crack tip.
\item The nodes which belongs to the elements totally cut by the crack, are enriched by and Heaviside function.
    $$h(x,y)=\begin{cases}1,&       for ~ ~ y\ge 0\\ -1,&       for~ ~ y\le 0\end{cases}$$

    \end{itemize}
\end{frame}
%%%%%%%%%%%%%%%%%%%%%%%%%%%%%%%%%%%%%%%%%%%%%%%%%%%%%%%%%%%%%%%%%%%%%%%%%%%%
\begin{frame}[t,fragile]{\ldots Extended Finite Element Method}
    \begin{itemize}
        \item In this work we have studied a recent method of modeling crack growth without re-meshing. The main advantage of this method is that the mesh is prepared without considering the existence of discontinuity.
        \item XFEM is based on the partition of unity method in which we enrich the classical finite element approximation to include the effects of singular discontinuous field around the crack.
            \begin{figure}[h]
                \centering
                \includegraphics[scale=.2]{enrich.png}
                \caption{XFEM enrichment strategy. Source: Abdelaziz, Y., Bendahane, K., \& Baraka, A. (2011). Extended Finite Element Modeling: Basic Review and Programming. Engineering, 3(07), 713.}
                \label{3}
            \end{figure}
    \end{itemize}
\end{frame}
%%%%%%%%%%%%%%%%%%%%%%%%%%%%%%%%%%%%%%%%%%%%%%%%%%%%%%%%%%%%%%%%%%%%%%%%
\begin{frame}[t,fragile]{Level Set Method}
    \begin{itemize}
        \item  The description of crack in extended finite element method is often described by the level-set method. A crack is described by two level-set methods as shown in figure below.
                     \begin{itemize}
        \item A normal level set, $\psi(x)$ which is the signed distance from the crack surface.
        \item A tangent level set, $\phi(x)$ which is the signed distance to the plane including the crack front and perpendicular to the crack surface.
        \end{itemize}
    \item To know which element should be enriched by which function we see the following
        \begin{itemize}
            \item If $\phi < 0 $ and $\psi_{min}\psi_{max}\leq 0$, then the crack cuts through the element and the nodes of the element are to be enriched with h(x).
            \item If in the element $\phi_{min}\phi{max}\leq 0$ and $\psi_{min}\psi{max}\leq 0 $, then the tip lies within that element and its nodes are to be enriched with $\gamma$.
        \end{itemize}

    \end{itemize}
\end{frame}
%%%%%%%%%%%%%%%%%%%%%%%%%%%%%%%%%%%%%%%%%%%%%%%%%%%%%%%%%%%%%%%%%%%%%%%%%%%%%%%
\begin{frame}
      \begin{figure}
                \centering
                \includegraphics[scale=.2]{levelset.jpg}
                \caption{Level Set Method}
                \label{4}
            \end{figure}
\end{frame}
%%%%%%%%%%%%%%%%%%%%%%%%%%%%%%%%%%%%%%%%%%%%%%%%%%%%%%%%%%%%%%%%%%%%%%%%%%%%
\end{document}


